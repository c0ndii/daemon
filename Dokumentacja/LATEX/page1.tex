{\bfseries{Temat 2 -\/ Demon synchronizujący dwa podkatalogi}}~\newline
 \mbox{[}12p.\mbox{]} Program który otrzymuje co najmniej dwa argumenty\+: ścieżkę źródłową oraz ścieżkę docelową . Jeżeli któraś ze ścieżek nie jest katalogiem program powraca natychmiast z komunikatem błędu. W przeciwnym wypadku staje się demonem. Demon wykonuje następujące czynności\+: śpi przez pięć minut (czas spania można zmieniać przy pomocy dodatkowego opcjonalnego argumentu), po czym po obudzeniu się porównuje katalog źródłowy z katalogiem docelowym. Pozycje które nie są zwykłymi plikami są ignorowane (np. katalogi i dowiązania symboliczne). Jeżeli demon (a) napotka na nowy plik w katalogu źródłowym, i tego pliku brak w katalogu docelowym lub (b) plik w katalogu źrodłowym ma późniejszą datę ostatniej modyfikacji demon wykonuje kopię pliku z katalogu źródłowego do katalogu docelowego -\/ ustawiając w katalogu docelowym datę modyfikacji tak aby przy kolejnym obudzeniu nie trzeba było wykonać kopii (chyba ze plik w katalogu źródłowym zostanie ponownie zmieniony). Jeżeli zaś odnajdzie plik w katalogu docelowym, którego nie ma w katalogu źródłowym to usuwa ten plik z katalogu docelowego. Możliwe jest również natychmiastowe obudzenie się demona poprzez wysłanie mu sygnału SIGUSR1. Wyczerpująca informacja o każdej akcji typu uśpienie/obudzenie się demona (naturalne lub w wyniku sygnału), wykonanie kopii lub usunięcie pliku jest przesłana do logu systemowego. Informacja ta powinna zawierać aktualną datę.~\newline
 a) \mbox{[}10p.\mbox{]} Dodatkowa opcja -\/R pozwalająca na rekurencyjną synchronizację katalogów (teraz pozycje będące katalogami nie są ignorowane). W szczególności jeżeli demon stwierdzi w katalogu docelowym ~\newline
 podkatalog którego brak w katalogu źródłowym powinien usunąć go wraz z zawartością.~\newline
 b) \mbox{[}12p.\mbox{]} W zależności od rozmiaru plików dla małych plików wykonywane jest kopiowanie przy pomocy read/write a w przypadku dużych przy pomocy mmap/write (plik źródłowy) zostaje zamapowany w całości w pamięci. Próg dzielący pliki małe od dużych może być przekazywany jako opcjonalny argument.~\newline
 Uwagi\+: (a) Wszelkie operacje na plikach i tworzenie demona należy wykonywać przy pomocy API Linuksa a nie standardowej biblioteki języka C (b) kopiowanie za każdym obudzeniem całego drzewa katalogów zostanie potraktowane jako poważny błąd (c) podobnie jak przerzucenie części zadań na shell systemowy (funkcja system). Demon którego głównym zadaniem jest utrzymywanie tej samej zawartości pomiędzy dwoma katalogami.~\newline
 Po podaniu katalogu źródłowego i docelowego demon rozpoczyna swoją pracę.~\newline
 Dodatkowo jest opcja aby program działał rekurencyjnie. ~\newline
 W celu sprawdzenia wykonanych podpunktów proszę udać się pod \mbox{\hyperlink{page2}{Wykonane podpunkty}}. 